% This is LLNCS.DEM the demonstration file of
% the LaTeX macro package from Springer-Verlag
% for Lecture Notes in Computer Science,
% version 2.4 for LaTeX2e as of 16. April 2010
%
\documentclass{llncs}
%
\usepackage{makeidx}  % allows for indexgeneration
%
\begin{document}
%
\title{Delta-theta intertrial coherence increases during task switching  in a BCI paradigm}
%
\titlerunning{ ITC in low frequencies in BCI}  % abbreviated title (for running head)
%                                     also used for the TOC unless
%                                     \toctitle is used
%
\author{Juan A. Barios\inst{1}*\and Santiago Ezquerro\inst{1} \and Arturo Bertomeu-Motos\inst{1}\and M. Nann \inst{2}\and S. Soekadar\inst{2}\and Nicolas Garcia-Araci\inst{1}}
%
\authorrunning{Juan A. Barios et al.} % abbreviated author list (for running head)
%
%%%% list of authors for the TOC (use if author list has to be modified)
\tocauthor{Juan A. Barios, Santiago Ezquerro, Arturo Bertomeu-Motos,  Irene Delegido, Abel Navarro-Arcas, J.M. Sabater-Navarro, Nicolas Garcia-Aracil}
%
\institute{Biomedical Neuroengineering research group (nBio), Systems Engineering and Automation Department of Miguel Hernandez University, Avda. de la Universidad s/n, 03202 Elche (Spain),\\
\email{jbarios@umh.es}, \texttt{http://nbio.umh.es/}
\and
University Hospital of Tübingen, Applied Neurotechnology Lab, Institute of Medical Psychology and Behavioral Neurobiology, Department of Psychiatry and Psychotherapy, Calwerstr. 14; D-72076 Tübingen; Germany,\\
\texttt{http://www.medizin.uni-tuebingen.de}
\thanks{Correspondence: jbarios@umh.es; Tel.: +34-965-222-505; Fax: +34-966-658-979}
}
\maketitle              % typeset the ntitle of the contribution




\begin{abstract}
Slow rythms are increasing their  importance in BCI paradigm. We report the presence of  significant changes in Delta-theta intertrial coherence increases during task switching  in a BCI paradigm\dots
\keywords{slow rythms, EEG, coherence, BCI}
\end{abstract}
%
\section{Introduction}
\subsection{This is a Second-Order Title}
\subsubsection{This is a Third-Order Title.} 
\paragraph{This is a Fourth-Order Title.}

\section{Material and Methods}
21\,$^{\circ}$C etc.,
Dr h.\,c.\,Rockefellar-Smith \dots
20,000\,km and Prof.\,Dr Mallory \dots
1950--1985 \dots
this -- written on a computer -- is now printed
$-30$\,K \dots

Italic ({\em <text>} better still \emph{<text>}) or, if necessary, {\bfseries boldface} should be used for emphasis.

Text with a footnote\footnote{The footnote is automatically numbered.} and text continues . . .

\begin{enumerate}
	\item First item
	\item Second item
	\begin{enumerate}
		\item First nested item
		\item Second nested item
	\end{enumerate}
	\item Third item
\end{enumerate}


\begin{figure}
	\vspace{2.5cm}
	\caption{This is the caption of the figure displaying a white
eagle and a white horse on a snow field}
\end{figure}

\begin{table}
	\caption{Critical $N$ values}
	\begin{center}
		\begin{tabular}{llllll}
			\hline\noalign{\smallskip}
			${\mathrm M}_\odot$ & $\beta_{0}$ & $T_{\mathrm c6}$ & $\gamma$
			& $N_{\mathrm{crit}}^{\mathrm L}$
			& $N_{\mathrm{crit}}^{\mathrm{Te}}$\\
			\noalign{\smallskip}
			\hline
			\noalign{\smallskip}
			30 & 0.82 & 38.4 & 35.7 & 154 & 320 \\
			60 & 0.67 & 42.1 & 34.7 & 138 & 340 \\
			120 & 0.52 & 45.1 & 34.0 & 124 & 370 \\
			\hline
		\end{tabular}
	\end{center}
\end{table}

Before continuing your text you need an empty line \dots

{\bfseries Bibliography:} \cite{clarke1982nonlinear}
%


%
% ---- Bibliography ----
%
\begin{thebibliography}{}
%
\bibitem{clarke1982nonlinear}
Clarke, F., Ekeland, I.:
Nonlinear oscillations and
boundary-value problems for Hamiltonian systems.
Arch. Rat. Mech. Anal. 78, 315--333 (1982)

\bibitem{2clar:eke:2}
Clarke, F., Ekeland, I.:
Solutions p\'{e}riodiques, du
p\'{e}riode donn\'{e}e, des \'{e}quations hamiltoniennes.
Note CRAS Paris 287, 1013--1015 (1978)

\bibitem{2mich:tar}
Michalek, R., Tarantello, G.:
Subharmonic solutions with prescribed minimal
period for nonautonomous Hamiltonian systems.
J. Diff. Eq. 72, 28--55 (1988)

\bibitem{2tar}
Tarantello, G.:
Subharmonic solutions for Hamiltonian
systems via a $\bbbz_{p}$ pseudoindex theory.
Annali di Matematica Pura (to appear)

\bibitem{2rab}
Rabinowitz, P.:
On subharmonic solutions of a Hamiltonian system.
Comm. Pure Appl. Math. 33, 609--633 (1980)

\end{thebibliography}

%\bibliography{biblio_sample}


\end{document}
